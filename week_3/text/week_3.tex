\documentclass[12pt]{article}
\usepackage[left=1in, right=1in]{geometry}

\usepackage{url}

%\usepackage{arydshln}

\usepackage{graphicx}

\usepackage{color}
\definecolor{light-gray}{gray}{0.30}

\usepackage{listings}
\lstset{
	frame=leftline,
	frameround=ttff,
	numbers=left,
	language=C++,
	showstringspaces=False,
	extendedchars=False,
	numberstyle=\footnotesize,
	basicstyle=\small\ttfamily,
	commentstyle=\color{light-gray}\slshape,
	belowskip=1.5em,
	aboveskip=1.5em,
	fontadjust,
	tabsize=4,		%added for proper tab alligning
	xleftmargin=0cm,
	xrightmargin=0cm
}


\newcommand{\desc}[1]{\textit{#1} \vspace{1em}}

\title{\itshape Exercises week 3}
%\subtitle{C/C++ course part II - }

\author{
	Klaas Isaac Bijlsma \\ s2394480
	\and
	David Vroom \\ s2309939
}

\date{\today}

\begin{document}
\maketitle

\section*{Exercise 20}
\desc{Learn the implications of using friends}

We used the following code,

\lstinputlisting[title=\texttt{binops/binops.h}]{../ex20/binops/binops.h}
\lstinputlisting[title=\texttt{addition/addition.h}]{../ex20/addition/addition.h}
\lstinputlisting[title=\texttt{subtraction/subtraction.h}]{../ex20/subtraction/subtraction.h}

\clearpage

\section*{Exercise 21}
\desc{Learn to implement a class hierarchy using friends in the final derived class}

We used the following code,

\lstinputlisting[title=\texttt{binops/binops.ih}]{../ex20/binops/binops.ih}
\lstinputlisting[title=\texttt{binops/binopsadd.cc}]{../ex20/binops/binopsadd.cc}
\lstinputlisting[title=\texttt{binops/binopssub.cc}]{../ex20/binops/binopssub.cc}
\lstinputlisting[title=\texttt{addition/addition.ih}]{../ex20/addition/addition.ih}
\lstinputlisting[title=\texttt{addition/operatoraddis1.cc}]{../ex20/addition/operatoraddis1.cc}
\lstinputlisting[title=\texttt{addition/operatoraddis2.cc}]{../ex20/addition/operatoraddis2.cc}
\lstinputlisting[title=\texttt{subtraction/subtraction.ih}]{../ex20/subtraction/subtraction.ih}
\lstinputlisting[title=\texttt{subtraction/operatorsubis1.cc}]{../ex20/subtraction/operatorsubis1.cc}
\lstinputlisting[title=\texttt{subtraction/operatorsubis2.cc}]{../ex20/subtraction/operatorsubis2.cc}


\clearpage

\section*{Exercise 22}
\desc{Learn to use a class hierarchy using friends in the final derived class}

We used the following code,

\lstinputlisting[title=\texttt{addition/operatoradd1.cc}]{../ex20/addition/operatoradd1.cc}
\lstinputlisting[title=\texttt{addition/operatoradd2.cc}]{../ex20/addition/operatoradd2.cc}
\lstinputlisting[title=\texttt{subtraction/operatorsub1.cc}]{../ex20/subtraction/operatorsub1.cc}
\lstinputlisting[title=\texttt{subtraction/operatorad2.cc}]{../ex20/subtraction/operatorsub2.cc}

\clearpage

\section*{Exercise 23}
\desc{Learn to use a class hierarchy using friends in the final derived class}

\clearpage

\section*{Exercise 24}
\desc{Learn to initialize \texttt{string} objects with \texttt{new}}
We used the following code,

\lstinputlisting[title=\texttt{main.ih}]{../ex24/main.ih}
\lstinputlisting[title=\texttt{main.cc}]{../ex24/main.cc}
\lstinputlisting[title=\texttt{factory.cc}]{../ex24/factory.cc}



\clearpage
\end{document}
