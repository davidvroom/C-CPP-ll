\documentclass[12pt]{article}
\usepackage[left=1in, right=1in]{geometry}

\usepackage{url}

%\usepackage{arydshln}

\usepackage{graphicx}

\usepackage{color}
\definecolor{light-gray}{gray}{0.30}

\usepackage{verbatim}

\usepackage{listings}
\lstset{
	frame=leftline,
	frameround=ttff,
	numbers=left,
	language=C++,
	showstringspaces=False,
	extendedchars=False,
	numberstyle=\footnotesize,
	basicstyle=\small\ttfamily,
	commentstyle=\color{light-gray}\slshape,
	belowskip=1.5em,
	aboveskip=1.5em,
	fontadjust,
	tabsize=4,		%added for proper tab alligning
	xleftmargin=0cm,
	xrightmargin=0cm
}


\newcommand{\desc}[1]{\textit{#1} \vspace{1em}}

\title{\itshape Exercises week 3}
%\subtitle{C/C++ course part II - }

\author{
	Klaas Isaac Bijlsma \\ s2394480
	\and
	David Vroom \\ s2309939
}

\date{\today}

\begin{document}
\maketitle

\section*{Exercise 20}
\desc{Learn the implications of using friends}

The exercise states that the classes \texttt{Addition} and \texttt{Subtraction}, which implement the binary addition and subtraction operators, are base classes of a class \texttt{Binops}. Furthermore, a class \texttt{Operations} implements the private functions \texttt{add} and \texttt{sub}, and inherits from \texttt{Binops}. Also, \texttt{Operations} declares \texttt{Binops} a friend class. 

\begin{itemize}
	\item
	An operator like \texttt{Addition::operator+=} has to call the private \texttt{add} member of \texttt{Operations}, but \texttt{Operations} does not declare \texttt{Addition} as a friend class. \\
	We solved this problem by defining a wrapper function \texttt{binopsAdd} in \texttt{Binops}, which calls \texttt{add}. This is allowed, since \texttt{Binops} is a friend class of \texttt{Operations}. Then, \texttt{Addition} is declared a friend class of \texttt{Binops}, which is therefore allowed to call \texttt{binopsAdd}, and by that it calls \texttt{add}. Ommiting the wrapper function does not work, since \texttt{Addition}, a friend class of a friend class of \texttt{Operations}, can not call \texttt{add}. The problem is handled in a similar way for \texttt{Subtraction}.  
	\item
	There is however a fundamental problem with this design. By defining the class hierarcy as stated in the exercise, multiple friendships among classes have to be defined. In this way, a strong coupling between the classes is obtained, which is undesirable. 

	\begin{comment}
	In fact, the class hierarchy should be the other way around, such that \texttt{Operations} is the base class defining the private \texttt{add} and \texttt{sub} functions, from which \texttt{Binops} inherits. \texttt{Addition} and \texttt{Subtraction} then inherit from \texttt{Binops}. In this way, no friend declarations causing extra coupling have to be defined. 
	\end{comment}
	\item
	Below we provide the class interfaces of \texttt{Binops}, \texttt{Addition} and \texttt{Subtraction}.
\end{itemize}

\lstinputlisting[title=\texttt{binops/binops.h}]{../ex20/binops/binops.h}
\lstinputlisting[title=\texttt{addition/addition.h}]{../ex20/addition/addition.h}
\lstinputlisting[title=\texttt{subtraction/subtraction.h}]{../ex20/subtraction/subtraction.h}

\clearpage

\section*{Exercise 21}
\desc{Learn to implement a class hierarchy using friends in the final derived class}

Below we provide the implementations of the classes \texttt{Binops}, \texttt{Addition} and \texttt{Subtraction}. (Not the free binary operators, see the next exercise for those.)

\lstinputlisting[title=\texttt{binops/binops.ih}]{../ex20/binops/binops.ih}
\lstinputlisting[title=\texttt{binops/binopsadd.cc}]{../ex20/binops/binopsadd.cc}
\lstinputlisting[title=\texttt{binops/binopssub.cc}]{../ex20/binops/binopssub.cc}
\lstinputlisting[title=\texttt{addition/addition.ih}]{../ex20/addition/addition.ih}
\lstinputlisting[title=\texttt{addition/operatoraddis1.cc}]{../ex20/addition/operatoraddis1.cc}
\lstinputlisting[title=\texttt{addition/operatoraddis2.cc}]{../ex20/addition/operatoraddis2.cc}
\lstinputlisting[title=\texttt{subtraction/subtraction.ih}]{../ex20/subtraction/subtraction.ih}
\lstinputlisting[title=\texttt{subtraction/operatorsubis1.cc}]{../ex20/subtraction/operatorsubis1.cc}
\lstinputlisting[title=\texttt{subtraction/operatorsubis2.cc}]{../ex20/subtraction/operatorsubis2.cc}


\clearpage

\section*{Exercise 22}
\desc{Learn to use a class hierarchy using friends in the final derived class}

Below we provide the remaining members of the classes \texttt{Binops}, \texttt{Addition} and \texttt{Subtraction}. The members \texttt{add} and \texttt{sub} are implemented inline, and therefore we provide the interface of \texttt{Operations} as well. After those listings, we provide a main function calling the different binary operators, and the output it produces. From that we see that the code works as intended. 

\lstinputlisting[title=\texttt{addition/operatoradd1.cc}]{../ex20/addition/operatoradd1.cc}
\lstinputlisting[title=\texttt{addition/operatoradd2.cc}]{../ex20/addition/operatoradd2.cc}
\lstinputlisting[title=\texttt{subtraction/operatorsub1.cc}]{../ex20/subtraction/operatorsub1.cc}
\lstinputlisting[title=\texttt{subtraction/operatorsub2.cc}]{../ex20/subtraction/operatorsub2.cc}
\lstinputlisting[title=\texttt{operations/operations.h}]{../ex20/operations/operations.h}

\lstinputlisting[title=\texttt{main.cc}]{../ex20/main.cc}
\lstinputlisting[title=\texttt{Output of main.cc},language=bash, numbers=none]{../ex20/output.txt}
\clearpage

\section*{Exercise 23}
\desc{Learn to use a class hierarchy using friends in the final derived class}

\clearpage

\section*{Exercise 24}
\desc{Learn to initialize \texttt{string} objects with \texttt{new}}

We used the following code,

\lstinputlisting[title=\texttt{main.ih}]{../ex24/main.ih}
\lstinputlisting[title=\texttt{main.cc}]{../ex24/main.cc}
\lstinputlisting[title=\texttt{factory.cc}]{../ex24/factory.cc}



\clearpage
\end{document}
