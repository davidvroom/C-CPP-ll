\documentclass[12pt]{article}
\usepackage[left=1in, right=1in]{geometry}

\usepackage{url}

%\usepackage{arydshln}

\usepackage{graphicx}

\usepackage{color}
\definecolor{light-gray}{gray}{0.30}

\usepackage{verbatim}

\usepackage{listings}
\lstset{
	frame=leftline,
	frameround=ttff,
	numbers=left,
	language=C++,
	showstringspaces=False,
	extendedchars=False,
	numberstyle=\footnotesize,
	basicstyle=\small\ttfamily,
	commentstyle=\color{light-gray}\slshape,
	belowskip=1.5em,
	aboveskip=1.5em,
	fontadjust,
	tabsize=4,		%added for proper tab alligning
	xleftmargin=0cm,
	xrightmargin=0cm
}


\newcommand{\desc}[1]{\textit{#1} \vspace{1em}}

\title{\itshape Exercises week 7 - Multi-threading I}

\author{
	Klaas Isaac Bijlsma \\ s2394480
	\and
	David Vroom \\ s2309939
}

\date{\today}

\begin{document}
\maketitle

\section*{Exercise 49}
\desc{Learn to apply basic multi-threading}

We used the following code.
\lstinputlisting[title=\texttt{main.ih}]{../ex49/main.ih}
\lstinputlisting[title=\texttt{waiting.cc}]{../ex49/waiting.cc}
\lstinputlisting[title=\texttt{main.cc}]{../ex49/main.cc}

\clearpage

\section*{Exercise 50}
\desc{Learn to perform time conversions}

We used the following code.
\lstinputlisting[title=\texttt{main.cc}]{../ex50/main.cc}

\clearpage

\section*{Exercise 51}
\desc{Learn to use the chrono/clock facilities}

We used the following code.
\lstinputlisting[title=\texttt{main.cc}]{../ex51/main.cc}

\clearpage

\section*{Exercise 52}
\desc{Learn to define a thread with objects that aren't functors}

We used the following code.

\lstinputlisting[title=\texttt{handler/handler.ih}]{../ex52/handler/handler.ih}    
\lstinputlisting[title=\texttt{handler/handler.h}]{../ex52/handler/handler.h}    
\lstinputlisting[title=\texttt{handler/shift.cc}]{../ex52/handler/shift.cc}      
\lstinputlisting[title=\texttt{main.cc}]{../ex52/main.cc}

\clearpage

\section*{Exercise 53}
\desc{Learn to design a simple producer/consumer program}


\clearpage

\end{document}
