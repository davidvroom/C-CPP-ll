\documentclass[12pt]{article}
\usepackage[left=1in, right=1in]{geometry}

\usepackage{url}

%\usepackage{arydshln}

\usepackage{graphicx}

\usepackage{color}
\definecolor{light-gray}{gray}{0.30}

\usepackage{listings}
\lstset{
	frame=leftline,
	frameround=ttff,
	numbers=left,
	language=C++,
	showstringspaces=False,
	extendedchars=False,
	numberstyle=\footnotesize,
	basicstyle=\small\ttfamily,
	commentstyle=\color{light-gray}\slshape,
	belowskip=1.5em,
	aboveskip=1.5em,
	fontadjust,
	tabsize=4,		%added for proper tab alligning
	xleftmargin=0cm,
	xrightmargin=0cm
}


\newcommand{\desc}[1]{\textit{#1} \vspace{1em}}

\title{\itshape Exercises week 2}

\author{
	Klaas Isaac Bijlsma \\ s2394480
	\and
	David Vroom \\ s2309939
}

\date{\today}

\begin{document}
\maketitle

\section*{Exercise 11}
\desc{}


\clearpage
\section*{Exercise 12}
\desc{Study the way \texttt{delete[]} works}

We used the following code,

\lstinputlisting[title=\texttt{maxfour/maxfour.h}]{../ex12/maxfour/maxfour.h}
\lstinputlisting[title=\texttt{maxfour/maxfour.ih}]{../ex12/maxfour/maxfour.ih}
\lstinputlisting[title=\texttt{maxfour/data.cc}]{../ex12/maxfour/data.cc}
\lstinputlisting[title=\texttt{maxfour/destructor.cc}]{../ex12/maxfour/destructor.cc}
\lstinputlisting[title=\texttt{main.ih}]{../ex12/main.ih}
\lstinputlisting[title=\texttt{main.cc}]{../ex12/main.cc}

\lstinputlisting[title=\texttt{Output of main.cc},language=bash, numbers=none]{../ex12/outputex12.txt}


\textbf{Explain why the solution is so simple}\\
The solution is so simple because when an exception is thrown during the construction of an array of 10 Maxfour objects, stack unwinding will destroy the already allocated objects. No explicit call of the destructor is needed. Furthermore we do not need to keep track of the already allocated objects. 

\clearpage
\section*{Exercise 13}
\desc{}

\clearpage
\section*{Exercise 14}
\desc{}


\clearpage
\section*{Exercise 15}
\desc{}

\clearpage
\section*{Exercise 16}
\desc{}


\clearpage
\section*{Exercise 17}
\desc{}


\clearpage
\section*{Exercise 18}
\desc{}


\clearpage
\section*{Exercise 19}
\desc{}


\clearpage
\end{document}
