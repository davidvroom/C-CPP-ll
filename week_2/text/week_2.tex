\documentclass[12pt]{article}
\usepackage[left=1in, right=1in]{geometry}

\usepackage{url}

%\usepackage{arydshln}

\usepackage{graphicx}

\usepackage{color}
\definecolor{light-gray}{gray}{0.30}

\usepackage{listings}
\lstset{
	frame=leftline,
	frameround=ttff,
	numbers=left,
	language=C++,
	showstringspaces=False,
	extendedchars=False,
	numberstyle=\footnotesize,
	basicstyle=\small\ttfamily,
	commentstyle=\color{light-gray}\slshape,
	belowskip=1.5em,
	aboveskip=1.5em,
	fontadjust,
	tabsize=4,		%added for proper tab alligning
	xleftmargin=0cm,
	xrightmargin=0cm
}


\newcommand{\desc}[1]{\textit{#1} \vspace{1em}}

\title{\itshape Exercises week 2}

\author{
	Klaas Isaac Bijlsma \\ s2394480
	\and
	David Vroom \\ s2309939
}

\date{\today}

\begin{document}
\maketitle

\section*{Exercise 11}
\desc{}


\clearpage
\section*{Exercise 12}
\desc{}

\clearpage
\section*{Exercise 13}
\desc{}

\clearpage
\section*{Exercise 14}
\desc{}


\clearpage
\section*{Exercise 15}
\desc{}

\clearpage
\section*{Exercise 16}
\desc{Learn how to end a program safely}

\textbf{How do you end a program in such a situation?}

In main wordt een object geconstruct en een functie aangeroepen die throwt. Deze functie doet vervolgens hetzelfde net als de functie daar weer in. In het diepste nested level wordt de throw operator daadwerkelijk aangeroepen, de exception gethrowd en vervolgens gerethrowd. Zodra de exception gethrowd wordt en de exception het try-block verlaat, wordt de destructor aangeroepen, net als in de levels daarboven.
Op deze manier worden alle constructed objects netjes vernietigd.\\

De volgende code verduidelijkt dit,

\lstinputlisting[title=\texttt{demo/demo.h}]{../ex16/demo/demo.h}
\lstinputlisting[title=\texttt{demo/demo.ih}]{../ex16/demo/demo.ih}
\lstinputlisting[title=\texttt{demo/demo.cc}]{../ex16/demo/demo.cc}
\lstinputlisting[title=\texttt{demo/destructor.cc}]{../ex16/demo/destructor.cc}
\lstinputlisting[title=\texttt{main.ih}]{../ex16/main.ih}
\lstinputlisting[title=\texttt{main.cc}]{../ex16/main.cc}
\lstinputlisting[title=\texttt{function1.cc}]{../ex16/function1.cc}
\lstinputlisting[title=\texttt{function2.cc}]{../ex16/function2.cc}
\lstinputlisting[title=\texttt{function3.cc}]{../ex16/function3.cc}

\clearpage
\section*{Exercise 17}
\desc{}


\clearpage
\section*{Exercise 18}
\desc{}


\clearpage
\section*{Exercise 19}
\desc{}


\clearpage
\end{document}
