\documentclass[12pt]{article}
\usepackage[left=1in, right=1in]{geometry}


\usepackage{url}

%\usepackage{arydshln}

\usepackage{graphicx}

\usepackage{color}
\definecolor{light-gray}{gray}{0.30}

\usepackage{listings}
\lstset{
	frame=leftline,
	frameround=ttff,
	numbers=left,
	language=C++,
	showstringspaces=False,
	extendedchars=False,
	numberstyle=\footnotesize,
	basicstyle=\small\ttfamily,
	commentstyle=\color{light-gray}\slshape,
	belowskip=1.5em,
	aboveskip=1.5em,
	fontadjust,
	tabsize=4,		%added for proper tab alligning
	xleftmargin=0cm,
	xrightmargin=0cm
}


\newcommand{\desc}[1]{\textit{#1} \vspace{1em}}

\title{\itshape Exercises week 1}

\author{
	Klaas Isaac Bijlsma \\ s2394480
	\and
	David Vroom \\ s2309939
}

\date{\today}

\begin{document}
\maketitle

\section*{Exercise 1}
\desc{Attain some familiarity with the way functions are selected from namespaces}

We used the following code,
%\lstinputlisting[title=\texttt{main.cc}]{../ex1/main.cpp}

\clearpage
\section*{Exercise 2}
\desc{Learn why streams can be used to determine the truth values of conditions, but not to assign values to bool variables.}

Note: The code given in the exercise is incomplete, and therefore won't compile even without the intended mistake. So first of all we state the following code as a starting point:

\lstinputlisting[title=\texttt{header.ih}]{../ex2/1/header.ih}
\lstinputlisting[title=\texttt{main.cc}]{../ex2/1/main.cc}
\lstinputlisting[title=\texttt{process.cc}]{../ex2/1/process.cc}
\lstinputlisting[title=\texttt{promptget.cc}]{../ex2/1/promptget.cc}

\subsection*{1.}
This code doesn't work, because \texttt{getline(in, str)} cannot be returned as a bool in \texttt{promptGet}. This is because the class \texttt{istream} defines \texttt{explicit operator bool() const}. This allows the compiler to only perform a conversion to a bool when this is explicitly required (as in a while statement), but not implicitly (as in the return statement above).

\subsection*{2.}
By changing \texttt{promptGet}'s body in the following way, the code does compile:
\lstinputlisting[title=\texttt{promptget.cc}]{../ex2/2/promptget.cc}

\subsection*{3.}
By changing \texttt{promptGet} (and the declaration in the internal header) in the following way, the code does compile:
\lstinputlisting[title=\texttt{promptget.cc}]{../ex2/3/promptget.cc}




\clearpage
\section*{Exercise 3}
\desc{Learn to implement index operators}

The Matrix class that is used here, is derived from the solutions of excercise 64.

We used the following code,
\lstinputlisting[title=\texttt{matrix/matrix.h}]{../ex5/matrix/matrix.h}

\clearpage
\section*{Exercise 4}
\desc{Learn to implement and spot opportunities for overloaded operators}

The header is shown in exercise 3, the implementations of the added functions are shown below:
\lstinputlisting[title=\texttt{matrix/add.cc}]{../ex7/matrix/add.cc}
\lstinputlisting[title=\texttt{matrix/operatoradd.cc}]{../ex7/matrix/operatoradd.cc}
\lstinputlisting[title=\texttt{matrix/operatoradd2.cc}]{../ex7/matrix/operatoradd2.cc}
\lstinputlisting[title=\texttt{matrix/operatorcompadd1.cc}]{../ex7/matrix/operatorcompadd1.cc}
\lstinputlisting[title=\texttt{matrix/operatorcompadd2.cc}]{../ex7/matrix/operatorcompadd2.cc}

\clearpage
\section*{Exercise 5}
\desc{Learn to insert/extract objects of your own class}

We used the following code,
\lstinputlisting[title=\texttt{matrix/extractcols.cc}]{../ex7/matrix/extractcols.cc}
\lstinputlisting[title=\texttt{matrix/extractrows.cc}]{../ex7/matrix/extractrows.cc}
\lstinputlisting[title=\texttt{matrix/functor1.cc}]{../ex7/matrix/functor1.cc}
\lstinputlisting[title=\texttt{matrix/functor2.cc}]{../ex7/matrix/functor2.cc}
\lstinputlisting[title=\texttt{matrix/functor3.cc}]{../ex7/matrix/functor3.cc}
\lstinputlisting[title=\texttt{matrix/operatorextract.cc}]{../ex7/matrix/operatorextract.cc}
\lstinputlisting[title=\texttt{matrix/operatorinsert.cc}]{../ex7/matrix/operatorinsert.cc}

\clearpage
\section*{Exercise 6}
\desc{}

\clearpage
\section*{Exercise 7}
\desc{Learn to implement and spot opportunities for overloaded operators}
\subsection*{1.}
The following two overloaded operators are added to compare two \texttt{Matrix} objects for (in)equality:
\lstinputlisting[title=\texttt{matrix/operatorequalto.cc}]{../ex7/matrix/operatorequalto.cc}
\lstinputlisting[title=\texttt{matrix/operatornotequalto.cc}]{../ex7/matrix/operatornotequalto.cc}
\subsection*{2.}
We modified the following code of the \texttt{Strings} class to facilitate comparing for (in)equality,
\lstinputlisting[title=\texttt{strings/strings.h}]{../ex7a/strings/strings.h}
\lstinputlisting[title=\texttt{strings/operatorequalto.cc}]{../ex7a/strings/operatorequalto.cc}
\lstinputlisting[title=\texttt{strings/operatornotequalto.cc}]{../ex7a/strings/operatornotequalto.cc}

\clearpage
\section*{Exercise 8}
\desc{}

\clearpage
\section*{Exercise 9}
\desc{}

\clearpage
\section*{Exercise 10}
\desc{}

\clearpage
\end{document}
