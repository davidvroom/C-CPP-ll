\documentclass[12pt]{article}

\usepackage{url}

%\usepackage{arydshln}

\usepackage{graphicx}

\usepackage{color}
\definecolor{light-gray}{gray}{0.30}

\usepackage{listings}
\lstset{
	frame=leftline,
	frameround=ttff,
	numbers=left,
	language=C++,
	showstringspaces=False,
	extendedchars=False,
	numberstyle=\footnotesize,
	basicstyle=\small\ttfamily,
	commentstyle=\color{light-gray}\slshape,
	belowskip=1.5em,
	aboveskip=1.5em,
	fontadjust
}

\usepackage[left=1.25in, right=1.25in]{geometry}

\newcommand{\desc}[1]{\textit{#1} \vspace{1em}}

\title{\itshape Exercises week 1}

\author{
	Klaas Isaac Bijlsma \\ s2394480
	\and
	David Vroom \\ s2309939
}

\date{\today}

\begin{document}
\maketitle

\section*{Exercise 1}
\desc{Attain some familiarity with the way functions are selected from namespaces}

We used the following code,
%\lstinputlisting[title=\texttt{main.cc}]{../ex1/main.cpp}

\clearpage
\section*{Exercise 2}
\desc{ziet ie dit?}

\clearpage
\section*{Exercise 3}
\desc{Learn to implement index operators}

The Matrix class that is used here, is derived from the solutions of excercise 64.

We used the following code,
\lstinputlisting[title=\texttt{matrix/matrix.h}]{../ex3/matrix/matrix.h}

\clearpage
\section*{Exercise 4}
\desc{}

\clearpage
\section*{Exercise 5}
\desc{}

\clearpage
\section*{Exercise 6}
\desc{}

\clearpage
\section*{Exercise 7}
\desc{}

\clearpage
\section*{Exercise 8}
\desc{}

\clearpage
\section*{Exercise 9}
\desc{}

\clearpage
\section*{Exercise 10}
\desc{}

\clearpage
\end{document}
