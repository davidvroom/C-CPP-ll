\documentclass[12pt]{article}
\usepackage[left=1in, right=1in]{geometry}

\usepackage{url}

%\usepackage{arydshln}

\usepackage{graphicx}

\usepackage{color}
\definecolor{light-gray}{gray}{0.30}

\usepackage{listings}
\lstset{
	frame=leftline,
	frameround=ttff,
	numbers=left,
	language=C++,
	showstringspaces=False,
	extendedchars=False,
	numberstyle=\footnotesize,
	basicstyle=\small\ttfamily,
	commentstyle=\color{light-gray}\slshape,
	belowskip=1.5em,
	aboveskip=1.5em,
	fontadjust,
	tabsize=4,		%added for proper tab alligning
	xleftmargin=0cm,
	xrightmargin=0cm
}


\newcommand{\desc}[1]{\textit{#1} \vspace{1em}}

\title{\itshape Exercises week 1, revision}
%\subtitle{C/C++ course part II - }

\author{
	Klaas Isaac Bijlsma \\ s2394480
	\and
	David Vroom \\ s2309939
}

\date{\today}

\begin{document}
\maketitle

\section*{Exercise 1}
\desc{Attain some familiarity with the way functions are selected from namespaces}

\clearpage
\section*{Exercise 2}
\desc{Learn why streams can be used to determine the truth values of conditions, but not to assign values to bool variables.}


\clearpage
\section*{Exercise 3}
\desc{Learn to implement index operators}

\clearpage
\section*{Exercise 4}
\desc{Learn to implement and spot opportunities for overloaded operators}


\clearpage
\section*{Exercise 5}
\desc{Learn to insert/extract objects of your own class}


\clearpage
\section*{Exercise 7}
\desc{Learn to implement and spot opportunities for overloaded operators}


\clearpage
\end{document}
