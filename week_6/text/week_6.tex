\documentclass[12pt]{article}
\usepackage[left=1in, right=1in]{geometry}

\usepackage{url}

%\usepackage{arydshln}

\usepackage{graphicx}

\usepackage{color}
\definecolor{light-gray}{gray}{0.30}

\usepackage{verbatim}

\usepackage{listings}
\lstset{
	frame=leftline,
	frameround=ttff,
	numbers=left,
	language=C++,
	showstringspaces=False,
	extendedchars=False,
	numberstyle=\footnotesize,
	basicstyle=\small\ttfamily,
	commentstyle=\color{light-gray}\slshape,
	belowskip=1.5em,
	aboveskip=1.5em,
	fontadjust,
	tabsize=4,		%added for proper tab alligning
	xleftmargin=0cm,
	xrightmargin=0cm
}


\newcommand{\desc}[1]{\textit{#1} \vspace{1em}}

\title{\itshape Exercises week 6}
%\subtitle{C/C++ course part II - }

\author{
	Klaas Isaac Bijlsma \\ s2394480
	\and
	David Vroom \\ s2309939
}

\date{\today}

\begin{document}
\maketitle

\section*{Exercise 42}
\desc{Learn to extract lines using generic algorithms into a container holding \texttt{string} objects, although \texttt{operator>>()} extracts strings}

\lstinputlisting[title=\texttt{main.cc}]{../ex42/main.cc}

\clearpage

\section*{Exercise 43}
\desc{Learn to use promotion with generic algorithms and predefined function objects when manipulating basic data types.}

\lstinputlisting[title=\texttt{main.cc}]{../ex43/main.cc}

\clearpage

\section*{Exercise 44}
\desc{Learn to recognize a situation where lambda functions may be used}

\lstinputlisting[title=\texttt{vstring/vstring.h}]{../ex44/vstring/vstring.h}
\lstinputlisting[title=\texttt{vstring/vstring.ih}]{../ex44/vstring/vstring.ih}
\lstinputlisting[title=\texttt{vstring/count.cc}]{../ex44/vstring/count.cc}
\lstinputlisting[title=\texttt{vstring/countchar.cc}]{../ex44/vstring/countchar.cc}
\lstinputlisting[title=\texttt{main.ih}]{../ex44/main.ih}
\lstinputlisting[title=\texttt{display.cc}]{../ex44/display.cc}
\lstinputlisting[title=\texttt{vowels.cc}]{../ex44/vowels.cc}
\lstinputlisting[title=\texttt{main.cc}]{../ex44/main.cc}

\clearpage

\section*{Exercise 45}
\desc{}

\clearpage

\section*{Exercise 46}
\desc{}

\clearpage

\section*{Exercise 47}
\desc{}

\clearpage

\section*{Exercise 48}
\desc{}

\clearpage

\end{document}
