\documentclass[12pt]{article}
\usepackage[left=1in, right=1in]{geometry}

\usepackage{url}

%\usepackage{arydshln}

\usepackage{graphicx}

\usepackage{color}
\definecolor{light-gray}{gray}{0.30}

\usepackage{verbatim}

\usepackage{listings}
\lstset{
	frame=leftline,
	frameround=ttff,
	numbers=left,
	language=C++,
	showstringspaces=False,
	extendedchars=False,
	numberstyle=\footnotesize,
	basicstyle=\small\ttfamily,
	commentstyle=\color{light-gray}\slshape,
	belowskip=1.5em,
	aboveskip=1.5em,
	fontadjust,
	tabsize=4,		%added for proper tab alligning
	xleftmargin=0cm,
	xrightmargin=0cm
}


\newcommand{\desc}[1]{\textit{#1} \vspace{1em}}

\title{\itshape Exercises week 4 - Polymorphism}
%\subtitle{C/C++ course part II - }

\author{
	Klaas Isaac Bijlsma \\ s2394480
	\and
	David Vroom \\ s2309939
}

\date{\today}

\begin{document}
\maketitle

\section*{Exercise 25}
\desc{Learn to construct an \texttt{ostream} class}

We constructed the class \texttt{Bistream}, which offers the same facilities as \texttt{ostream}, but inserts its information into two files, whose \texttt{ofstream}-objects are passed to this class's constructor. A second class \texttt{BiStreamBuffer} is made and used. We used the following code,

\lstinputlisting[title=\texttt{bistream/bistream.h}]{../ex25/bistream/bistream.h}
\lstinputlisting[title=\texttt{bistream/bistream.ih}]{../ex25/bistream/bistream.ih}
\lstinputlisting[title=\texttt{bistream/bistream1.cc}]{../ex25/bistream/bistream1.cc}

\lstinputlisting[title=\texttt{bistreambuffer/bistreambuffer.h}]{../ex25/bistreambuffer/bistreambuffer.h}
\lstinputlisting[title=\texttt{bistreambuffer/bistreambuffer.ih}]{../ex25/bistreambuffer/bistreambuffer.ih}
\lstinputlisting[title=\texttt{bistreambuffer/bistreambuffer1.cc}]{../ex25/bistreambuffer/bistreambuffer1.cc}
\lstinputlisting[title=\texttt{bistreambuffer/overflow.cc}]{../ex25/bistreambuffer/overflow.cc}
\clearpage

\section*{Exercise 26}
\desc{Learn to design a streambuf reading from file descriptors}

We designed the class \texttt{IFdStreambuf}, whose objects may be used as a \texttt{streambuf} of \texttt{istream} objects to allow extractions from an already open file descriptor. We used the following code,

\lstinputlisting[title=\texttt{ifdstreambuf.h}]{../ex26/ifdstreambuf/ifdstreambuf.h}
\lstinputlisting[title=\texttt{ifdstreambuf.ih}]{../ex26/ifdstreambuf/ifdstreambuf.ih}
\lstinputlisting[title=\texttt{close.cc}]{../ex26/ifdstreambuf/close.cc}
\lstinputlisting[title=\texttt{destructor.cc}]{../ex26/ifdstreambuf/destructor.cc}
\lstinputlisting[title=\texttt{ifdstreambuf1.cc}]{../ex26/ifdstreambuf/ifdstreambuf1.cc}
\lstinputlisting[title=\texttt{ifdstreambuf2.cc}]{../ex26/ifdstreambuf/ifdstreambuf2.cc}
\lstinputlisting[title=\texttt{open.cc}]{../ex26/ifdstreambuf/open.cc}
\lstinputlisting[title=\texttt{underflow.cc}]{../ex26/ifdstreambuf/underflow.cc}
\lstinputlisting[title=\texttt{xsgetn.cc}]{../ex26/ifdstreambuf/xsgetn.cc}



\clearpage

\section*{Exercise 27}
\desc{Learn to design a streambuf writing to file descriptors}

We designed the class \texttt{OFdStreambuf}, whose objects may be used as a \texttt{streambuf} of \texttt{ostream} objects to allow insertions into an file descriptor. We used the following code,

\lstinputlisting[title=\texttt{ofdstreambuf.h}]{../ex27/ofdstreambuf/ofdstreambuf.h}
\lstinputlisting[title=\texttt{ofdstreambuf.ih}]{../ex27/ofdstreambuf/ofdstreambuf.ih}
\lstinputlisting[title=\texttt{close.cc}]{../ex27/ofdstreambuf/close.cc}
\lstinputlisting[title=\texttt{destructor.cc}]{../ex27/ofdstreambuf/destructor.cc}
\lstinputlisting[title=\texttt{ofdstreambuf1.cc}]{../ex27/ofdstreambuf/ofdstreambuf1.cc}
\lstinputlisting[title=\texttt{ofdstreambuf2.cc}]{../ex27/ofdstreambuf/ofdstreambuf2.cc}
\lstinputlisting[title=\texttt{open.cc}]{../ex27/ofdstreambuf/open.cc}
\lstinputlisting[title=\texttt{overflow.cc}]{../ex27/ofdstreambuf/overflow.cc}
\lstinputlisting[title=\texttt{sync.cc}]{../ex27/ofdstreambuf/sync.cc}


\clearpage

\section*{Exercise 28}
\desc{Learn to design streams}

We designed \texttt{IFdStream} and \texttt{OFdStream}, which are \texttt{istream} and \texttt{ostream} objects, respectively, reading from and writing to streams. We also made a main function that copies information entered at the keyboard to the screen. We used the following code,

\lstinputlisting[title=\texttt{ifdstream/ifdstream.h}]{../ex28/ifdstream/ifdstream.h}
\lstinputlisting[title=\texttt{ifdstream/ifdstream.ih}]{../ex28/ifdstream/ifdstream.ih}
\lstinputlisting[title=\texttt{ifdstream/ifdstream.cc}]{../ex28/ifdstream/ifdstream.cc}

\lstinputlisting[title=\texttt{ofdstream/ofdstream.h}]{../ex28/ofdstream/ofdstream.h}
\lstinputlisting[title=\texttt{ofdstream/ofdstream.ih}]{../ex28/ofdstream/ofdstream.ih}
\lstinputlisting[title=\texttt{ofdstream/ofdstream.cc}]{../ex28/ofdstream/ofdstream.cc}

\lstinputlisting[title=\texttt{main.cc}]{../ex28/main.cc}


\clearpage

\section*{Exercise 29}
\desc{}

\clearpage

\section*{Exercise 30}
\desc{}

\clearpage

\section*{Exercise 31}
\desc{}

\clearpage

\end{document}
